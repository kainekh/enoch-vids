
\documentclass{article}
\usepackage[pdftex]{graphicx}
\usepackage[T1]{fontenc}
\usepackage[utf8]{inputenc}
\usepackage{pslatex}
\usepackage[greek.polutoniko,hebrew,english]{babel}
\usepackage{cjhebrew}
\usepackage[backend=biber, style=sbl]{biblatex}
\addbibresource{$BIB}

\title{Introduction to I Enoch}
\author{Kenneth Gardner}

\begin{document}

\maketitle

\section{Pseudepigrapha}

Pseudepigrapha literally means "lying writing."
We use it to denote works that exist in Second Temple Judaism and which are not in a Protestant, Orthodox, or Roman Catholic Old Testaments.
By contrast, we use the title "apocrypha" to denote works in the Chalcedonian Orthodox and Roman Catholic Old Testaments but not in the Hebrew Bible.
Some example works within the "pseudepigrapha" category are, I Enoch, the Book of Jubilees, Joseph and Aseneth, 3 Ezra, and the Assumption of Moses.
It derives the name because in many cases, works were published in another person's name.
The various books of Enoch, for instance, were not written by Enoch.
Neither were the books of Ezra written by Ezra.

The best modern parallel to pseudepigrapha is the practice of ghost writing.
People hire writers to write anonymously in their name to argue certain points.
If a politician releases a book, for instance, it is not unlikely that the politician barely touched it.
Ghost writing is not identical to pseudepigrapha, but it is a form of pseudonymity.
It is, in effect, a "lying name," but it is also ethical in both antiquity and modernity.\footnote{
  Almost every Greek student has read \emph{Lysias I}.
}
I mention politicians, because both rivals and the average man actively looks for hypocrisy in them, and are so eager that, even when they find hypocrisy, they're quick to make some up.
However, pseudonymity very rarely becomes a bone of contention.
Why?
It's accepted as ethical, but if we can accept that, then it's not a stretch to consider pseudepigraphic authorship could be in other cultures.

And there are good reasons to believe that they considered it ethical.
In 104.9-10 we find:
\begin{quote}
Do not err in your hearts or lie,\newline
or alter the words of truth,\newline
or falsify the words of the Holy One,\newline
or give praise to your errors.\newline
For it is not to righteousness\newline
all your lies and all your errors\newline
lead, but to great sin\newline
And now I know this mystery,\newline
that the sinners will alter and\newline
copy the words of truth\newline
and pervert many and lie and\newline
invent fabrications and write\newline
books in their own names.\newline
Would that they would write all\newline
words in truth and neither\newline
remove nor alter these words.\newline
but write in truth all\newline
that I testify to them.
\end{quote}
The translation in the Hermeneia translation is unique, and their commentary does not offer any enlightenment.
However this wording raises the question:
Did the author of this portion of I Enoch think pseudepigrapha was a moral obligation when writing?
It is certainly possible, but it's certainly not clear enough to fight over.

However, this begs the question, why would they consider it ethical?
We often cannot answer that about our own culture and have to guess at it, so we're going to be forced to guess in an ancient and foreign one.
So we are starting out with a "best guess" approach to this question.
I reckon that getting that out up front is necessary for honest disclosure.
That said, what are our guesses?

Bruce Metzger lists several possible reasons.\footcite{BMetzger1972}
\begin{itemize}
  \item Financial gain
  \item Malice
  \item Respect
  \item Modesty
\end{itemize}

I don't think that financial gain is a viable option, and neither did Metzger.
The ancients simply didn't have a concept of copyright.
Once an author had released a document, it was forever beyond his control, and he would receive no royalties from copies.
Unless an author's name was on it, he could hope for wealth from it.
This sort of decadence simply isn't the sort of thing they'd fallen into, even if it sounds strange since it seems self-evident in our culture.
As such, financial gain is simply not a plausible motive.

Outright malice is a possibility for ancient pseudepigrapha.
We've all had someone lie about us.
Maliciously misrepresenting people in court, for instance, is the point of the "do not bear false witness."
Lying and putting words in the mouths of others predates writing.
This possibility is, therefore, \emph{quite} believable.
We also have examples from antiquity.
The article I've been referencing referenced the example of Anaximenes, who forged a document to slander his opponents.\footcite[6]{BMetzger1972}
The Apostle Paul warned that Christians shouldn't be troubled by some letter purporting to be from them; someone was forging letters in his name.\footnote{2 Thess 2.2}

However malice cannot be a motive for I Enoch's authors.
They are not trying to malign Enoch in any way.
Rather they portray him as a scribe of scribes, someone people can treat like Moses.\footnote{Jubilees later regards I Enoch on par with Moses.}
Cultivating this sort of respect is mutually exclusive with malice.
We can safely disregard that out of the gate.

Now the idea of respect accords most with what we see in I Enoch.
There is absolutely no question the authors respected Enoch for the same reason we know that it wasn't out of malice.
The book practically oozes respect and awe for Enoch.
However, it begs the question: was it written out of respect for him?
At no place does it say, ``We do this because we respect Enoch.''
Therefore, it would be an assumption to say that respect was \emph{the} reason they wrote it.
However, it strikes me as very safe that, if it weren't the reason for pseudepigraphic authorship, it is, at the very least, one of the reasons Enoch was chosen as the target.

Modesty is another possible answer.
The quote I gave above from 104.9-10 is a fairly good indication it might have been.
If they believed writing on one's own name is a vice or, at the very least, hubris, then it follows that they would write either anonymously or pseudepigraphically.
If we combine that with the evident respect, it can form a complete and coherent model.
The problem is, we cannot demonstrate this.
The translation is idiosyncratic, and I cannot address the Geez fragments.
Even if it weren't, the passage is hardly a ``This is why I did it'' statement.
It would still be speculation.
Another problem would be: ``Why does modesty morph into pseudepigrapha?''
For that, I have nothing.
It's easy to see that such a stance could lead to pseudepigrapha, but it is not at all clear that how it did.
The same silence that makes all these statements speculation also makes it so that ``how we got there'' is even more speculative.
I'm not willing to entertain it.

Metzger offers several other motives, but these four stick out to me as being worthy of consideration.
I lean toward modesty and respect together, but, without a way to demonstrate it, it's little more than a brute assertion.
The article is a worthy read, and it's far from the only scholarly article on the subject.
I selected it, however, due to how concise it is and how well structured.

Another, more important point, is that I think pseudepigrapha is, in many ways, a category error.
I think it is a bad idea to refer to it as ``pseudepigrapha.''
``Second Temple literature'' seems to be gaining steam, and that is much better.
I find ``pseudepigrapha'' simply too misleading most of the time I see it deployed.
Joseph and Aseneth is a pseudepigraphic work that makes no pretense of authorship.
Not all the apocalypses designate an author or use first person language either.

Likewise, some portions of what we call ``apocrypha'' are manifestly pseudepigraphic.
The ``Wisdom of Solomon'' is not written by Solomon.\footnote{
	Technically, it doesn't say it is, but it makes a strong allusion to it.
}
In the agreed canon, ``Daniel'' was, at the very least, has portions not written by Daniel and good arguments none of it was.
Is it ``pseudepigrapha?''

In a similar problem, the using ``apocrypha'' to designate works not in the Protestant Bible but in the Roman Catholic or Chalcedonian Orthodox Bibles is inadequate.
What do we do with Ethiopian Jews and Christians?
It's in \emph{their} Bible, and their Bible is old enough it predates any pretense to a closed canon.
In a similar manner, where we would look at the apocrypha and ask how it was treated over the centuries by people who used it, I'm hard-pressed to find a single person who accords them the same respect, and I don't even have access to that information myself.
It's a problem I would love to see redressed.
As such, from this point on, I'm going to just refer to the genre as ``Second Temple Literature.''
This is much clearer and more respectful of those separated bodies.\footnote{
	I am aware that some people in the past expanded apocrypha to include the Ethiopians, but it would be too confusing to do here.
}

\section{Apocalypse}

I Enoch is generally classified as an apocalyptic work.
Following, there are five elements that we can use to describe apocalyptic.
\begin{enumerate}
  \item Cosmic dualism
  \item Earthly mirrors heavenly
  \item Cosmic journey
  \item Angelic mediation
  \item Temporal dualism\footcite{37-40}{GNickelsburg1}
\end{enumerate}

Good and evil forces both are governing the spiritual world and in perpetual conflict.
The two ``kingdoms'' are opposed to each other and in constant war.
This war is reflected on the earth (hence, both cosmic dualism and the earthly mirroring).
Beasts portray nations and leaders.
Natural phenomena in visions reflect those in others.

Probably the clearest example of both of these is Daniel 7.
In chapter 7 Daniel saw a succession of four empires.
Among these grew up a particularly arrogant horn who ``spoke arrogantly,'' which in the context was blasphemed.
The Ancient of Days arose and judged the nations.
The arrogant horn itself was destroyed with fire.

In this short passage, we see both the cosmic dualism and the mirror effect.
For the dualism, these peoples are opposed to the peoples of God.
Their princes make war with those of the people of God.
Later, in 10.20, Michael, the prince over Israel would make war with the prince of Persia.

There is undisputed conflict being portrayed.
However unlike, say, the conflict between Ahura Mazda and Ahriman,\footnote{
	These are the good and evil forces of Zoroastrianism.
}
there is no substantial fight.
The Ancient of days is victorious.
We know there is conflict, but it simply says that thrones were set up, and the horn is destroyed.
The horn itself is the culmination of several empires.
Each one supplanted the other, and the little horn supplanted them.
Thus history is symbolized by what we see in the cosmic realm.

Another example would be the dragon in Revelation 12.
The dragon, which is the great dragon is the ancient serpent.
It appears under several guises in Scripture.
It is the serpent of Eden,\footnote{
	I know this is an unpopular opinion today, but it is the historical Christian interpretation.
}
Leviathan the twisting serpent,\footnote{
  Leviathan is an ancient chaos monster with seven heads that \textsc{Yhwh} slays on the last day and in creation. e.g. Psalm 73/74.13-14, Is. 27.1
}
Satan, and several other names.
To lift a line from Job, he is the ``king of all the sons of pride.''\footnote{Job 41.34}
He attacks the woman, Israel, who is bringing about the Messiah, and when he fails, he takes it out on her progeny.
Not coincidentally, right afterward the beasts of Revelation 13 arise, which denote facets of the Roman Empire (and, I would say, successive empires).

Related to this, the temporal dualism ties into this struggle.
There is a pristine period prior to the fall that was lost.
However the real contrast is always that of the present evil age and a pristine coming age.
God will, in the future, decisively judge and crush the forces of evil and usher in the Age to Come.

The cosmic journey is best exemplified by John's Revelation.
He is ``taken up'' on the ``Lord's Day,'' an especially apt time for conflict, as it kicks off the liturgical themes of the book.
He then goes on a tour of heaven where he sees these sights.
He sees the throne of God, the abyss, various lands, and so on.
Enoch's journey is far more elaborate, but it's the same themes.

``Angelic mediation'' denotes a few different themes.
First, and most obvious, the angels intercede about people on earth.
Angels stand at the altar of God, and they pray and present the prayers of men at the altar.
The textbook biblical example of this is Revelation 8.3-4.
There the angel presents a burning incense, which the text labels ``the prayers of the holy ones.''

Beyond that, however, there is another element to this theme.
Generally the seer needs what he sees interpreted.
The angel, or some other heavenly being interprets what the seer sees.
The heavenly being, however, is not God generally.
There is a celestial hierarchy, to borrow from Ps. Dionysius, places angels between men and God.
God speaks to the angel, and he to us.
This pattern holds even after Christ's sacrifice, as Revelation shows.

There are reasons to doubt some of this.
First, apoacalyptic was so diverse that it doesn't always fit the themes above.
Second, there's no indication the authors thought, ``OK, let's write an apocalyptic text.''
I suspect the genre ``apocalyptic'' is inherently anachronistic and, thus, as misleading as it is revealing.\footnote{
  This is a concern I have with many ancient genres and structures we find.
  We humans are extremely good at finding patterns that don't exist or weren't intended.
  The patterns we find are useful to us to understand the text, but they may not always be aboriginal and thus may not lead us to a contextual understanding.
  Healthy suspicion is always warranted with attempts to isolate the correct reading as the original as if we can divorce the text from readers any more than we can the authors, its transmission, or its form.
}
For an idea of the ambiguity inherent in the term ``apocalyptic,'' the discussion in Amy Richter's book:
\begin{quote}
  Already, in the identification of two trends in the research into apocalyptic themes in Matthew's Gospel, the confusion and disagreement over what scholas meant by ``apocalyptic'' is apparent.
  Were ``apocalyptic'' and ``eschatologial'' largely interchangeable?
  Before going further, it is necessary to pause and brieflymention significant atempts to define the words and phrases used in studies of Matthew and apocalyptic motifs.
  This brief excursus will provide an opportunity to see how evolving definitions of ``apocalypse,'' a``apocalypticism,'' and ``apocalyptic eschatology'' opened the doors for scholars to add more motifs outside of thsoe related to end-times and final judgment and participate in the second trend, which includes more material with its scope.\footcite[280]{Richter2012}
\end{quote}
The above wouldn't make sense if it were easily definable.
She spends her first chapter discussing what apocalyptic is just so she can articulate how to tie it into Matthew.
The definition of even so familiar a term as ``apocalypse'' is simply \emph{not} self-evident or even always clear.


\section{Structure}

Enoch is several books mashed together.
The five major sections are:
\begin{enumerate}
  \item Book of Watchers (BoW)
  \item Book of Parables (BoP)
  \item Book of Luminaries (BoL)
  \item Book of Visions (BoV)
  \item Epistle of Enoch (BoE)\footnote{
	  I know, it doesn't abbreviate to that, but if I keep the theme from the preceding it'll at least be easy to remember.
	  }
\end{enumerate}

The BoW includes the prologue, the unholy pact, and the giant wars.
It occupies chapters 1-36 and is the oldest portion of the text.
It probably dates to the third century BC.

The book of Watchers begins with a theophany.
After the theophany, it includes a call to righteousness by way of an appeal to nature.
It then recounts the sin of the angels, Enoch's ascent, and the giant wars.

The BoP is the latest portion of the text.
It is in chapters 37-71.
It is incredibly hard to date, because it makes no hard links to history and because no portion has been preserved in anything but Geez.
It includes a more detailed account of Enoch's divinization.
Perhaps the most interesting thing for Christians is just how much the language about Enoch as the Son of Man mirrors that found in the Synoptics.
However, even this is muddled with linguistic and textual problems.

The BoL is an astronomical book.
It occupies chapters 83-90.
It is a bit odd in that there is almost no reference to the sin of the Watchers in it.
There is also no reference to a final judgment in it.
It purports to explain the exact order of the heavens.

The BoV is in chapters 83-90.
It is broken up into two large visions.
The first, the Animal Apocalypse, recounts history (with a strong eye toward the Jewish history) by way of animals.
Then there is another that recounts history as a series of weeks.

BoE takes up chapters 92-105.
It is part of the testamental genre.
The genre takes its lead from Genesis' stories about the blessings of the patriarchs.
Each patriarch blesses his sons, and when he does so, he includes blessings and prophecies that prophecy about future roles or what God intends.
Other examples of the genre are the various Testaments of the Twelve Patriarchs, which come to us in varying states of preservation.
This portion also provides the literary hook about how the text comes down to us.
Tertullian, for instance, took it at face value and asserted Noah had preserved it.\footcite[1.3.1-3]{SchaffATertAppWom}

There are various miscellaneous smaller portions as well.
Two of note are the Birth of Noah, 106-107, and a Final Book of Enoch (108).
The former provides a defense against the allegation that Noah himself was a result of co-mingling with the Watchers.
The latter is a later addition that attempts to summarize and interpret the preceding.

\section{State of the Text}

What the preceding tells us is that Enoch has gone through many periods of redaction.
Since the BoP is found in none of the DSSs, then it is an open question whether the book even existed there.
That question, alone, raises the question of whether Ethiopic Enoch was the same as that that the Apostle Jude cites as Scripture.
It may well be in a very different form.

Another issue is that the BoL in the DSSs indicates a quite different shape.
Size estimates vary, but it may have been substantially larger the text that we have.\footcite[357]{GNickelsburg2}
The fragmentary portions do not give us enough information to properly line up its size.
If the estimate is correct, then the BoL would have been too large to be transmitted with rest of the text.
The Greek text we have shows transparent signs of abridgment (something I will discuss when we get there).

With just those data, it's worth stopping for a second and considering just how different the book could be.
The BoL and BoP take up probably half, maybe more, of the text.
If they were added after the time of St. Jude, or added in stages, then our book is quite different today.
This sort of redaction introduces quite a bit of uncertainty.
Individual books themselves show evidence of redaction that is sometimes easy, sometimes hard, to get around.
For instance, the Gr.BoW uses ``Watchers,'' \textgreek{ἐγρήγορος} \emph{egrēgros}, almost exclusively for evil heavenly beings.
The good ones are typically called ``angels.''
However, we have DSS fragments of the BoW, and in those places we have ``Watchers,'' and it matches the Aramaic formulae for things like ``Watchers, holy ones.''\footcite[141]{GNickelsburg1}

Another issue isn't just the physical state of the text, but the very words we must deal with.
The book is preserved in its entirity only in Ethiopic.
The Ethiopic is most likely a translation of the Greek text, but we cannot rule out a translation of the Aramaic.
The Greek translates the Aramaic edition.
After finding the DSS fragments, we can be relatively confident that most, or all, of it was composed in Aramaic.
This means that, for most of I Enoch, we are dealing with a translation (English), of a translation (Geez), of a translation (Greek), of an original (Aramaic) that is a chimeric union of several sources.
That should inspire uncertainty just as much as it invites speculation, and, together, they form an unhealthy, and, yes, pleasurable intellectual admixture.

\section{Context}

It has another implication, though.
We simply cannot say for certain much about the context.
In most cases, we cannot even discern the motivations or social context as anything but shadows.
We are permanently chained to the wall watching shadows, and if we look too long at the context, we will mistake the shadows for the reality.

What I mean is that we have at least five large sections written in different contexts.
They were linked together in stages, the context of which we do not know.
We do not know much about what changes were made when the texts were joined together.
We cannot date with certainty large swaths of it, and what we can, we still don't know the social motivations, strata, or broader theological context.
We must speculate about the context of the book, but we must simultaneously realize that our context is as much a necessary fiction as it is a reality.
We cannot interpret without some context, and so where we do not have it, we must guess, and there is a lot of guesswork.

Let's add another layer of confusion to the mix.
Apocalyptic symbolism is dense and easily misread.
Suppose I said I saw a fleet of boats, and at the helm of the largest boat, I saw a great bull.
This bull went to and fro about the earth, grinding everything else to dust.
Then it fought a ship with a lion on it, but the lion destroyed the bull's boats.
After that, the lion began to grind the lands into dust, and as it did one of the areas it had laid waste gave rise to a great eagle.
This eagle threw the lion's paw off of its land, and then fought once more.
After that, a great double-headed eagle arose, and everywhere it landed blood filled the land, death and moaning were heard.
The lion and the eagle joined forces with a bear and slew the double-headed eagle.
Then the bear turned on the lion and the eagle, and so the eagle and the lion fought.
The fought on all lands, and they turned the world into ash.
Eventually the eagle slew the bear, but then the eagle grew fat.
Its feathers grew frayed and was slow, and over the horizon a panda arose and began devouring the people.

What did I just say?
We know, and it is easy for us.
However, if you don't know the history, then the symbolism is basically impenetrable.
I used world events for that, and, likely, in a thousand years scholars could easily decode that.
What would have been like, though, if I used symbolism like that to denote social movements in the USA today, and that most of those records were lost?
Then the book was translated multiple times, edited, and joined with other books?
The context and meaning would be incredibly difficult to retrieve.

And that is where I Enoch is at.
The text is in bad shape.
It is a symbolic book, but it's not always clear what is symbolic, or even what it refers to.
Is the BoW targeting the Diadachoi\footnote{The successors to Alexander the Great} or to the priesthood taking foreign wives?
Does it rely on the story of Prometheus, or do they both assume a similar story?
There are many questions like that that arise in the text, and a goodly number of them profoundly change how we read the text.
It isn't clear at all.
It's also deeply influential on the NT, and so it's good to study.
Taken together, it means that there's a lot of fun to be had.

\section{Bibliography}

\printbibliography

\end{document}
