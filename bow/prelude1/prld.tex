\documentclass{beamer}
\usepackage[T1]{fontenc}
\usepackage[utf8]{inputenc}
\usepackage{pslatex}
\usepackage[greek.polutoniko,hebrew,english]{babel}
\usepackage{cjhebrew}
\usetheme{Berkeley}
\usecolortheme{albatross}

\title{Prelude to the Book of Watchers}
\author{Kenneth Gardner}

\begin{document}

\maketitle

\begin{frame}
  The prelude draws on wisdom literature, such as Proverbs.
\end{frame}

\begin{frame}
  This wisdom literature makes appeals to nature to establish its point.
  Nature is treated as unchanging and a place to look to see the goodness of God.
\end{frame}

\begin{frame}
  The prelude to the BoW calls on people to look upon the unchanging aspects of nature.
  The stars and planets do not vary in their courses.
  This invariability is a sign that they obey God.
\end{frame}

\begin{frame}
  People, by contrast, have changed their ways.
  They have abandoned the way of God and become corrupt.
\end{frame}

\begin{frame}
  This brings a theophany.
  A theophany is the revelation of the deity upon earth.
  It is typically accompanied by language like ``the mountains melt'' and ``the land gives way.''
\end{frame}

\begin{frame}
  It is also called the Day of \textsc{Lord} in the Old Testament.
  It is associated with judgment and deliverance, but it is a fearful day for all.
\end{frame}

\begin{frame}
  An important note is that this language is often hyperbolic.
  There have been many moments called the ``Day of the \textsc{Lord}'' in the OT.
  We have to judge by context whether a judgment is the Great Judgment, whether it is one of the others, or more than one.
\end{frame}

\begin{frame}
  Some examples:\pause
  \begin{itemize}
	\item Lam 2.22 regards the destruction of Jersusalem this way.\pause
	\item Amos 5.18 promises God's judgment on Israel and Judah.\pause
	\item Zeph. 1 has a universal one overthrowing creation akin to the Flood.\pause
	\item In Jer 46.10, it is the defeat of Egypt.
  \end{itemize}
\end{frame}

\begin{frame}
  Theophanies that are not called the Day of the \textsc{Lord} include:\pause
  \begin{itemize}
	\item Christ's baptism Mt. 3\pause
	\item Isaiah's call Is. 6\pause
	\item The Transfiguration Mt 17\pause
	\item Jacob's Ladder Genesis 28\pause
	\item Paul's Damascene vision Acts 9
  \end{itemize}
\end{frame}

\begin{frame}
  The archetypal theophany we should think of is the theophany in Exodus 19, even if it's not always in the background.
  The judgments of God are aimed at establishing his order in the hearts of men and to reveal himself.
\end{frame}

\begin{frame}
  Even though Enoch uses language like the Book of Watchers, the Law itself does not feature prominently.
  There are two basic approaches:\pause
  \begin{itemize}
	\item I Enoch has some opposition to Moses and so minimizes him.\pause
	\item I Enoch is more universal in scope and is set before the Law, when the real Enoch would have lived.
  \end{itemize}
\end{frame}

\section{Superscription 1:1}

\begin{frame}
  \begin{quote}
	The words of the blessing of Enoch, wherewith he blessed the elect and righteous, who will be living in the day of tribulation, when all the wicked and godless are to be removed.
  \end{quote}
\end{frame}

\begin{frame}
  ``Wheels within wheels.''\\
  Dune used this as a reference to plans within plans.
  Enoch introduces the prelude which introduces the Book of Watchers here.
\end{frame}

\begin{frame}
  ``Words of blessing of Enoch wherewith he blessed the elect and righteous''\\
  Enoch aims for his words to be a blessing and means of salvation first.
\end{frame}

\begin{frame}
  The phrase ``elect and righteous'' occurs many times.
  It denotes the chosen lot that will inherit God's blessing and whom God has chosen as his lot.
\end{frame}

\begin{frame}
  It hearkens to Dt. 33.1
  \begin{quote}
	This is the blessing with which Moses, the man of God blessed the children of Israel before his death.
  \end{quote}
\end{frame}

\begin{frame}
  This serves as both an allusion and mimesis.
  Mimesis is where someone deliberate imitates the style or structure of something else for rhetorical or narrative effect.
\end{frame}

\begin{frame}
  The point of an allusion is to recall the passage.
  It is not aimed at referencing an isolated verse the way children are taught Bible memory verses.
\end{frame}

\begin{frame}
  \begin{itemize}
	\item Dt. 33.1 follows the Song of Moses.\pause
	\item In Dt. 32, Moses sings of how God chose Israel as his inheritance and eschewed the nations that had been alotted (8-9).\pause
	\item Then Israel rebelled and served territorial spirits (v. 17) over the universal God.\pause
	\item The song ends with a promise that God will avenge his people on those he allowed to overcome them (42).\pause
	\item This ultimately leads to salvation for the nations that they should rejoice (43).
  \end{itemize}
\end{frame}

\begin{frame}
  The judgment includes Moses, for Moses had sinned.
  Moses, the center of righteous humanity in the Pentateuch is forbidden to enter the Promised Land.
  He could only see at afar off, and so the OT showed glimpses of Christ, but we saw they could only look at from afar.
  We, in our turn, await the final destruction of evil.
\end{frame}

\begin{frame}
  It is at this point Moses pronounces his blessing.
  God will rise up, and he will march on the nations and their gods.
  He sets in order the priesthood and social order they would need to combat the nations and the gods.
\end{frame}

\begin{frame}
  This is also the theme of my favorite Psalm: Ps. 67/68.
  God stands, and he makes war on the fatty mountains of the gods, leads them captive, and he delivers humanity from their clutches when Christ arises on Pascha.
\end{frame}

\begin{frame}
  It is here we are introduced to Enoch:
  \begin{quote}
	And he took up his parable and said—Enoch a righteous man, whose eyes were opened by God, saw the vision of the Holy One in the heavens, whichthe angels showed me, and from them I heard everything, and from them I understood as I saw, but not for this generation, but for a remote one which is for to come.
  \end{quote}
\end{frame}

\begin{frame}
  There is an allusion to Balaam in Numbers 24 here, vv. 3-4, 15-17.
\end{frame}

\begin{frame}
  The context is that Balak has asked Balaam to curse the children of Israel.
  He said he would speak only what God gave him, and he blessed them.
\end{frame}

\begin{frame}
  Enoch is a universal prophet.
  He predates the Mosaic covenant.
\end{frame}

\begin{frame}
  Like Balaam, he sees a conflict between the elect of God and the nations.
\end{frame}

\begin{frame}
  Alluding to this passage sets the tone for the book.
  The Watchers will descend, and God will send a ``last days'' upon them.
\end{frame}

\begin{frame}
  There is an inscription from Deir `Alla in which Balaam receives a vision from the gods and prophecies.
  It is possible Enoch imitates this.
\end{frame}

\begin{frame}
  If it refers to Deir `Alla, then it is subversive.\pause
  \begin{itemize}
	\item Balaam was prominent.\pause
	\item Enoch was a universal prophet and a gentile.
  \end{itemize}
\end{frame}

\begin{frame}
  Was Enoch evangelistic?\pause
  \begin{itemize}
	\item I Enoch has strong parallels to Balaam.\pause
	\item The parallels are generally positive(at least for Enoch).\pause
	\item Enoch uses a Babylonian calender for its base.\pause
	\item It doesn't feature the Mosaic covenant very prominently.\pause
	\item In the BoV, a key feature is that the gentiles become followers of the one God and there is one people.
  \end{itemize}
\end{frame}

\begin{frame}
  If Evangelistic, I Enoch is intended to subvert gentile calenders, posit a reverse theogoony, and call people to repentence (both Jew and gentile).
\end{frame}

\end{document}
