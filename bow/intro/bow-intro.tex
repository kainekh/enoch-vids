\documentclass{beamer}
\usepackage[T1]{fontenc}
\usepackage[utf8]{inputenc}
\usepackage{pslatex}
\usepackage[greek.polutoniko,hebrew,english]{babel}
\usepackage{cjhebrew}
\usetheme{Berkeley}
\usecolortheme{albatross}

\title{Book of Watchers: Introduction}
\author{Kenneth Gardner}

\begin{document}

\maketitle

\section{Book of Watchers background}

\begin{frame}
  \huge{Book of Watchers background}
\end{frame}

\begin{frame}
  The Book of Watchers is the oldest book in I Enoch.
  It was probably written about the 3rd century BC.
\end{frame}

\begin{frame}
  It was written in Aramaic.
  Fragments have been found in the Dead Sea Scrolls.
  There is also a \emph{pešer} on it.
\end{frame}

\begin{frame}
  A \emph{pešer} is a commentary on the Bible.
  The \emph{pešer} is on fragments 4Q180-181.
\end{frame}

\begin{frame}
  It is part of the rewritten Bible tradition.
  This tradition takes OT texts and expands upon them.
  This text focuses heavily on Genesis 6.
\end{frame}

\begin{frame}
  For a cause of writing, the Book of Watchers may be focusing on\pause
  \begin{itemize}
	\item A corrupt priesthood that had taken foreign wives or illicit wives\pause
	\item A reworking of the Prometheus myth\pause
	\item An attack on the Diadochoi\pause
	\item An account of something having to do with the heavenly court
  \end{itemize}
\end{frame}

\begin{frame}
  Related, it occurred to me that there may be a relation to Homer on account of a talk given by Gregory Nagy.
  I have not worked this out or been able to properly research it, so take it with a salt shaker of skepticism.\pause
  \begin{itemize}
	\item The Book of Watchers portrays the giants as waging warfare and engulfing the land in violence.\pause
	\item The Iliad portrays the Achaeans making war on Troy and undermining a more dignified civilization\pause
	\item The giants pollute the very creation so that it us unmade\pause
	\item Gregory Nagy argues that the heroes in Homer rendered themselves and those around them unclean.\pause
	\item Genesis 6 may provide a background where the giants build classical civilization.\pause
	\item Homer is the foundation of Gr\ae co-Roman civilization, and it traces itself back to demigods.
  \end{itemize}
\end{frame}

\begin{frame}
  These are several points of contact, but I need to think on it more.
  It would make sense for the BoW, if it is taking shots at Greek culture, to take shots at Homer.
\end{frame}

\section{Rewritten Bible}

\begin{frame}
  \huge{Rewritten Bible}
\end{frame}

\begin{frame}
  The rewritten Bible genre receives biblical tales, then adds extra details to interpret them.
  It also brings the stories up to then contemporary interests.
\end{frame}

\begin{frame}
  We have the same genre.
  There are many novels sold in our stores that set themselves in the Bible.
  They insert details to make them seem more like what the author and audience believe, practice, and/or expect.
  They rarely have much historical reality under them.
\end{frame}

\begin{frame}
  I Enoch is \emph{not} a primary witness to Genesis.
  It represents an early tradition, but it is a secondary witness.
\end{frame}

\begin{frame}
  It represents three generations of giants (7.2):\pause
  \begin{itemize}
	\item Giants\pause
	\item Nephilim\pause
	\item Elioud
  \end{itemize}
\end{frame}

\begin{frame}
  This only makes sense if we are dealing with a misunderstanding.
  What, exactly, are ``Elioud?''
  How does the plain word ``giants'' relate to a Greek transliteration of an Aramaic transliteration of a Hebrew appropriation of an Aramaic word meaning ``giant?''
  This only makes sense if the BoW is interpreting Genesis, and either the original author or the translator didn't understand the text.
\end{frame}

\begin{frame}
  \begin{itemize}
	\item Primary witnesses would not have this sort of confusion.\pause
	\item A secondary witness struggling with the meaning of a difficult passage would.\pause
	\item The Book of Watchers, thus, cannot definitively settle the meaning of Gen. 6.1-4
  \end{itemize}
\end{frame}

\section{Genesis 6}

\begin{frame}
  \huge{Genesis 6.1-4}
\end{frame}

\begin{frame}
  Since I Enoch is \emph{not} a primary witness to Genesis, there are a few things we should keep in mind.\pause
  \begin{itemize}
	\item Genesis should be interpreted on its own with I Enoch as one supporting line of evidence for the interpretation.\pause
	\item When we look at Genesis for I Enoch, we should look principally for ambiguities and clues that give rise to the interpretation.\pause
	  \begin{itemize}
		\item We should pay special heed to areas where Genesis, neighboring cultures, and I Enoch intersect.\pause
		\item We should not be concerned with the ``correct'' reading of Genesis when we are concerned with how the BoW interprets it; we should be concerned with how it was \emph{read}.
	  \end{itemize}
  \end{itemize}
\end{frame}

\begin{frame}
  \begin{columns}
	\column{.5\textwidth}
	\footnotesize{And it happened that when humanity began to multiply on the face of the land, and daughters were born to them,
	and the sons of God saw the daughters of humanity that they were good to behold, and they took for themselves wives from all that they chose,
	and \textsc{Yhwh} said, my Spirit will not remain in humanity indefinitely in that they are flesh, and their days will be one hundred and twenty years;
	the giants were on the land in those days, and afterwards as well whenever the sons of God went into the daughters of humanity, and there were born to them those warriors, which were the men of the name from long ago. (\textbf{MT})}
	\column{.5\textwidth}
	\footnotesize{And it happened that when humans began to become numerous upon the land, and daughters were born to them
	that the sons of God seeing the daughters of humans, that they were beautiful, took to themselves wives from all that they chose,
	and the Lord God said, In no way will my spirit dwell in these humans for the age, because they are flesh. Their days will be a hundred twenty years.
	Now the giants were upon the land in those days and after those when the sons of God were going to the daughters of humans, and they bore for them. These were the giants, who were from the age the men of the name. (\textbf{LXX})}
  \end{columns}
\end{frame}

\begin{frame}
  There are several ambiguities with this text.
  I included the LXX, because it helps to show the BoW reading wasn't unique for the period, and because it is canonical for my Church.
\end{frame}

\begin{frame}
  The first thing to note is that the verb ``began'' in Hebrew has connotations of ``to corrupt'' in certain contexts.
  While it carries the primary meaning of ``begin'' indisputably, it is possible ancient readers read a subtext into it of ``corrupt.''
  Double-entendres like this based on consonants are not uncommon in Hebrew.
\end{frame}

\begin{frame}
  The meaning of ``sons of God'' is clear.
  These are heavenly beings that, in some fashion, copulate with the daughters of men.
\end{frame}

\begin{frame}
  God saw what was happening, and he said his Spirit would \cjRL{lo'--yAdwon}.
  The LXX renders it ``abide.''
  It might also mean ``contend'' or ``restrain.''
  It is conspicuously absent from I En. 6, but is it present at 10.9-10?
  Or is it omitted entirely?
\end{frame}

\begin{frame}
  Following the previous question, is the phrase \cjRL{bA'AdAM} and \cjRL{b*:+sag*aM hw*' bA,sAr}, ``in humans\ldots [they] are also flesh,'' referring to humans or to giants?
  If to humans, why would the author say that ``these humans\ldots are also flesh?''
\end{frame}

\begin{frame}
  We know from the grammar construction that the giants in v. 4 are connected to v. 3.
  However, it has not yet declared this action evil.
  Were the warriors, the \emph{gibborim}, the giants, or were they a second class of being?
\end{frame}

\begin{frame}
  We know the warriors are the ``men of the Name,'' but were they good or evil?
  ``The name'' can be used for fame, but it is elsewhere used as a standin for \textsc{Yhwh}.
  Could these warriors be servants of \textsc{Yhwh}, or does the name merely refer to fame?
\end{frame}

\begin{frame}
  These are \emph{some} of the questions we must consider.
  We'll answer them as time goes on.
  Our concern is not what Genesis meant.
  Our concern is what the author of I Enoch \emph{thought} it meant.
  For more information on these questions, Archie T. Wright has an excellent book, \emph{The Origin of Evil Spirits}.
  He goes into detail on these questions.
\end{frame}

\section{Structure}

\begin{frame}
  \huge{The Book of Watchers' Structure}
\end{frame}

\begin{frame}
  \begin{itemize}
	\item Prelude (1-5)\pause
	\item Rebellion (6-8)\pause
	\item Heavenly intercession (9-11)\pause
	\item Enoch's intercession (12-16)\pause
	\item Enoch's cosmic journey (17-36)
  \end{itemize}
\end{frame}

\begin{frame}
  The prelude begins with a theophany.
  God is going to thunder forth from Zion to judge the world.
  He will have his angels at his side to pronounce judgment on the wicked.
\end{frame}

\begin{frame}
  The prelude then appeals to the works of heaven as an example.
  They are changeless, so should humans be changeless in their steadfastness to God's commands.
  However humans have not.
  This leads to a verdict that wicked people will curse their days and be a curse.
\end{frame}

\begin{frame}
  The rebellion section opens with the Shemihazah calling on the other rebel watchers to join him.
  He secures this with an oath.
  They then enter into women to sire children.
\end{frame}

\begin{frame}
  After it recounts their horrors, it gives a list of watchers and their names.
  These names are organized to correlate with the secrets they reveal.
\end{frame}

\begin{frame}
  It then cuts to the throne room of God.
  There the archangels, as members of God's heavenly court, intercede on behalf of humanity.
  God hears and gives the four archangels tasks to secure the cleansing of the world.
\end{frame}

\begin{frame}
  After this, Enoch is introduced and interacts with the fallen watchers.
  God sends him there, and then they ask him to intercede on their behalf.
  Their request is denied.
\end{frame}

\begin{frame}
  Enoch then begins his journey and sees the ends of the world.
  He sees the various folds of Sheol and how the dead are detained as well as the place of punishment for the fallen watchers.
  He sees things like the Tree of life, Jerusalem, and the paradise.
\end{frame}

\end{document}
