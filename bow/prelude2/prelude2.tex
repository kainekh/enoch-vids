\documentclass{beamer}
\usepackage[T1]{fontenc}
\usepackage[utf8]{inputenc}
\usepackage{pslatex}
\usepackage[greek.polutoniko,hebrew,english]{babel}
\usepackage{cjhebrew}
\usetheme{Berkeley}
\usecolortheme{albatross}

\title{BoW Prelude 1.3-9}
\author{Kenneth Gardner}

\begin{document}

\maketitle

\section{Meter}

\begin{frame}
  Semitic poetry operates by allusions:\pause
  \begin{itemize}
	\item Blessed is the man who walks not in the ways of the ungodly
	\item Nor sits in the seat of the pestilent
  \end{itemize}
\end{frame}

\begin{frame}
  The structure of the poem:
  \begin{itemize}
	\item[a] The Holy Great One will come forth from His dwelling,
	\item[a] And the eternal God will tread upon the earth, (even) on Mount Sinai,
	\item[b] And appear from His camp
	\item[b] And appear in the strength of His might from the heaven of heavens.
	\item[c] And all shall be smitten with fear,
	\item[c] And the Watchers shall quake,
	\item[d] And great fear and trembling shall seize them unto the ends of the earth.
	\item[d] And the high mountains shall be shaken.
  \end{itemize}
\end{frame}

\begin{frame}
  The meter in 1.3-9 may be the original meter to the oracles.
  Black asks, ``Was this the original (poetic) structure of Enoch's oracular discourse?''
\end{frame}

\begin{frame}
  Texts were not necessarily marked off for meter in ancient poetry.
  You need to be able to recognize the poetry by meter and not by presentation.
\end{frame}

\section{Allusions}

\begin{frame}
  The primary allusions for 1.3-9 are:\pause
  \begin{itemize}
	\item Dt. 33
	\item Mic 1.3-4
	\item Jer 25.30-31
  \end{itemize}
\end{frame}

\begin{frame}
  Dt. 33\pause
  \begin{itemize}
	\item God comes from Sinai\pause
	\item comes with ``ten thousands of holy ones''\pause
	\item v 27 refers to the dwelling of the righteous being the Eternal God.\pause
	\item God comes forth in judgment out of love for his people
  \end{itemize}
\end{frame}

\begin{frame}
  Jer 25.30-31\pause
  \begin{itemize}
	\item God comes from his dwelling\pause
	\item He comes to judge all and all flesh
  \end{itemize}
\end{frame}

\begin{frame}
  Mic. 1.3-4\pause
  \begin{itemize}
	\item God's coming\pause
	\item tread\pause
	\item Melt like wax
  \end{itemize}
\end{frame}

\begin{frame}
  Is 26.21:\pause
  \begin{itemize}
	\item God is coming forth to punish\pause
	\item reference to blood parallels coming imagery about giants shedding blood
  \end{itemize}
\end{frame}

\begin{frame}
  Ps 67/68:\pause
  \begin{itemize}
	\item ``melts like wax''\pause
	\item judgment over fallen spiritual powers\pause
	\item The earth ``quakes'' before him\pause
	\item Sentiment survives liturgically in a version of Ps. 67/68
  \end{itemize}
\end{frame}

\begin{frame}
  The Easter hymn goes:
  \begin{quote}
	Let God arise and his enemies be scattered; and let those that hate him flee from his face\\
	\emph{chorus} Christ is risen from the dead, trampling down death by death, and to those in the tombs bestowing life (3x)\\
	As smoke vanishes, let them vanish as wax melts before the fire\\
	\emph{chorus}\\
	So shall the sinners perish from the face of God and the righteous be glad\\
	\emph{chorus}\\
	This is the day that the Lord has made, and we shall rejoice and be glad in it
  \end{quote}
\end{frame}

\begin{frame}
  A related Easter hymn is:
  \begin{quote}
	\emph{chorus} Arise O God and judge the earth, for you shall have an inheritance among all the nations\\
	God stood in the congregation of the gods, and in the midst he shall stand out among gods\\
	\emph{chorus}\\
	How long will ye judge unrighteously and accept the person of sinners\\
  \end{quote}
\end{frame}

\begin{frame}
  These Easter connections bring the subject to Jude 14-5:
  \begin{quote}
	    And Enoch also, the seventh from Adam, prophesied of these, saying,
        Behold, the Lord cometh with ten thousands of his saints,
		To execute judgment upon all, and to convince all that are ungodly among
        them of all their ungodly deeds which they have ungodly committed, and
        of all their hard speeches which ungodly sinners have spoken against
        him.
  \end{quote}
\end{frame}

\begin{frame}
  Jude intends to refer to God's coming return to judge the wicked, among whom he counts the proto-Gnostic's.
\end{frame}

\begin{frame}
  This passage may also be alluded to in Mt. 25.31, but more likely Dan 7.13.
\end{frame}

\section{Commentary}

\begin{frame}
  The passage presents God as coming forth in judgment due to the sins of the world.
\end{frame}

\begin{frame}
  The theophany passage has God coming forth in military fashion.
  He comes forth at the head of armies to conquer his enemies.
\end{frame}

\begin{frame}
  Compare this with the Baal Cycle where Baal goes forth and subjugates a list of towns.
  It is not intended an exhaustive list but symbolizing the inhabited world.
\end{frame}

\begin{frame}
  While not a list, this intends the same thing.
  God is coming forth upon all flesh and all creation, spiritual and physical.
\end{frame}

\begin{frame}
  In the same vein, the ``tread'' is not merely ``walking'' but ``trampling.''
  God is going to run roughshod over his enemies in the same way God tramples on the high places in Micah 1.3.
\end{frame}

\begin{frame}
  This is a holy war between God and the watchers.
\end{frame}

\begin{frame}
  The watchers ``will quake'' because of the might of this force and because they are doomed.
\end{frame}

\begin{frame}
  Both the Charles translation and Matthew Black's have that those at the end of the earth shall quake.
  However the Hermeneia Enoch has them singing.
\end{frame}

\begin{frame}
  If we take the Hermeneia reading, they interpret the song as a ``song of lament'' from the watchers.
  It is reasonable to read it this way.
\end{frame}

\begin{frame}
  The song may also refer to the oppressed people singing a song of praise to God for their liberation (cf. Is 24.16).
  It is the cry of those whose blood has been spilt that precipitates God's judgment (9.2), and they make suit in Hades later as well (I En 22.5, where ``the man'' is Abel and the watchers represent Cain's rebellion).
\end{frame}

\begin{frame}
  I do not think the image can be used in the Origenist sense of Philippians 2.11 where all creation ``confesses'' that is ``eagerly agrees'' or even, in LXX usage, ``praises'' \textgreek{>exomolog'ew}.
  A reconciliation of the watchers is simply not in view in I Enoch.
\end{frame}

\begin{frame}
  There is another variant that \emph{does} indicate an eventual conversion of the watchers, because a Greek translation reads ``and will believe.''
  The Greek ``believe'' indicates loyalty, and it is not used of beings that are disobedient for the timeframe of the verb.
\end{frame}

\begin{frame}
  The language of creation breaking and melting is fairly typical of theophanies (cf. Mic 1.3-4, Ps. 67/8.8, Ps. 97.5, Nah 1.5).
  It is used to emphasize the fear of God's judgment and, I suspect, a forward look to the last judgment.
\end{frame}

\begin{frame}
  The sole arguments, off the top of my head, for this are that the watchers beg Enoch to petition for mercy (13) and that there is a duration for how long Azazel will be bound (10.5).
\end{frame}

\begin{frame}
  In the case of the former, they are answered ``You shall have no peace,'' and in the latter the imprisonment is followed by being cast into the fire of the age to come.
\end{frame}

\begin{frame}
  Concerning this sense, Black says, ``Commentators tend to accept the reading ``shall believe''; e.g.
  Milik, 145, thinks the text contained an allusion to the `Origenistic conception of the conversion of evil spirits.''
  But Black himself rejects this reading (I Enoch, 107).
\end{frame}

\begin{frame}
  If \textgreek{piste'usousin} were original, then it would indicate that.
  We lack the Aramaic.
  I doubt the reading because, if it were so prominently in the opening, I would expect the idea to be represented later fairly clearly.
  Instead parallels to passages like 13.3 give no room for it.
\end{frame}

\begin{frame}
  It is particularly apt as a description just before the Flood.
\end{frame}

\begin{frame}
  Enoch and Genesis portray the Flood as a form of decreation.
\end{frame}

\begin{frame}
  After the judgment of the wicked, God will ``make peace'' with the righteous, who will ``belong to God.''
\end{frame}

\begin{frame}
  Black raises the possibility that ``belong to God'' may have an underlying reading of ``become sons of God.''
  This would indicate a process of divinization as found in I Enoch 71 and in the beatitudes.
  The language is favored by the language of ``light shining upon them.''
\end{frame}

\begin{frame}
  The ``make peace with'' indicates either that there was a divide or that he will establish peace and just rule with them.
\end{frame}

\begin{frame}
  This passage closes with a promise that God is coming with his armies to execute judgment on all, which Jude quotes.
\end{frame}

\section{The Theophany in Christianity}

\begin{frame}
  This judgment motif is appropriated by the NT for the second coming and for the crucifixion.
\end{frame}

\begin{frame}
  The motif does \emph{not} have to come from Enoch.
  It exists outside of it, but it is a touch point.
\end{frame}

\begin{frame}
  It is easy to see the motif in both Jude 14-15 and Mt. 25.31.
\end{frame}

\begin{frame}
  However, it is present in the crucifixion in general.
  Ephesians 4 places the judgment of the wicked powers at the cross.
\end{frame}

\begin{frame}
  In this passage Paul quotes Ps. 67/8.18, and our passage alludes to the same Psalm.
\end{frame}

\begin{frame}
  Paul's image is that Christ has defeated the spiritual powers and given us gifts.
  In Col 2.15, he presents Christ as having not only defeated them on the cross but led them in a Triumph.
\end{frame}

\begin{frame}
  The Triumph was a vile Roman practice where the general would be paraded with captives before the city of Rome.
  They would paint their faces red to become an image of Jupiter.
  At the of the procession they would ritually strangle the captives they humiliated before Jupiter's statues.
\end{frame}

\begin{frame}
  Paul appropriates it to give some sense of just how badly the gods of the nations have been humiliated and defeated.
\end{frame}

\begin{frame}
  The theme is also present in I Cor. 2.8 where he mentions that had they known, they would not have crucified the ``Lord of Glory.''
\end{frame}

\begin{frame}
  For this reason, images of the crucifixion in the middle ages often had ``the King of glory'' written on the cross instead of the message in the Synoptics.
\end{frame}

\begin{frame}
  Lastly, Ps. 23/24 was a Psalm for a Temple context.
  It subverts imagery about Baal, who descends to take his throne from Mot and has a series of similar dialogues.
  He says, for instance, ``O Gods, lift up your heads.“
\end{frame}

\begin{frame}
  However in the Psalm, God is assaulting the palace of death and overthrowing Baal.
  For this reason, the passage is used in the Liturgy for Holy Saturday, and it is used in the consecration of new churches.
\end{frame}

\begin{frame}
  For us, the \textsc{Yhwh}'s military march is the march of Christ on Hades.
  In the same way, the theophany in Enoch becomes for us the theophany on the Christ.
  Even the Second Coming is more aftershocks of Christ's victory there.
\end{frame}

\end{document}
